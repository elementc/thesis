%========================================
\section{Long Sequence Testing in Yeager}

%--------------------------------------
\begin{frame}
  \frametitle{Testers Write Based On The System's States}
  \begin{itemize}
    \item Test scripts often are developed emulating the system's underlying state model.
    \item Implied state model is significantly simplified compared to a formal FSM specification.
  \end{itemize}
\end{frame}

\begin{frame}
  \frametitle{Context: What Simplified State Models Don't Capture}
  If we could capture just the states and transitions implied by the test scripts being written, we'd still be missing:
  \begin{itemize}
    \item Input typed into the program
    \item Data the program read from some external source
    \item Overheating CPUs, full disks, cosmic rays, etc.
  \end{itemize}
  These ``Simplified State Models'' might still be useful.
\end{frame}

\begin{frame}
  \frametitle{Simplified State Models Can Be Represented As Directed Multigraphs}
  \begin{itemize}
    \item System states are vertexes, or nodes.
    \item Test functions are edges, connecting an in-node to an out-node.
    \item Each edge connects one in-node to one out-node, however
    \begin{itemize}
      \item a given function might work as a transition to an out-node from multiple compatible in-nodes.
      \item This behavior is a byproduct of convenicence features in the software under test, like having a logout button on every page.
      \item For brevity's sake, treat a list of in-nodes on an edge's definition as a separate edge definition for each listed in-node.
    \end{itemize}
  \end{itemize}
\end{frame}

\begin{frame}
  \frametitle{Random Walks: Generating New Test Plans Automatically}
  Given one of these simplified state models represented as a graph, and a source of random numbers, automatically generating test plans is straightforward.
  \begin{itemize}
    \item For a given node, the current state, from the set of nodes
    \item Gather all of the edges, the transition functions, which have that state as their from-node
    \item Select and execute one of the gathered functions
    \item The selected function's to-node becomes the new current state
    \item Repeat until some planned condition is met or execution of a selected function is not possible
  \end{itemize}
\end{frame}

\begin{frame}
  \frametitle{What Bugs Look Like From A Modeling Perspective}
  \begin{itemize}
    \item Bugs manifest as nodes which the model says should be reachable, but execution cannot successfully reach.
    \item Such occurrences might be bugs in the software.
    \item Such occurrences might be bugs in the tester's model.
  \end{itemize}
\end{frame}

\begin{frame}
  \frametitle{Prior Art: Model Based Testing}
  \begin{itemize}
    \item Jonathan Jacky, in Radiation Oncology, of the University of Washington, made an excellent Python model-based tester called PyModel.
    \item PyModel consumes a handcrafted model.
    \item It can emit a test plan that covers the whole model.
    \item It can emit a test plan that takes a random, should-be valid walk of the software under test.
  \end{itemize}
\end{frame}

\begin{frame}
  \frametitle{Weaknesses in PyModel}
  \begin{itemize}
    \item It requires a handcrafted model in a finicky domain-specific language.
    \begin{itemize}
      \item Not Plain Old Python.
    \end{itemize}
    \item It is difficult to connect to test execution.
    \item It requires a lot of time to get running.
  \end{itemize}
\end{frame}

%----------------
\subsection{Usage}

\begin{frame}
  \frametitle{What Is Yeager?}
  \begin{itemize}
    \item Python version 3 module
    \item Annotate funtions indicating that they cause a state transition
    \item Infers a state model
    \item Can take a random walk on that model
    \begin{itemize}
      \item Can terminate random walks under selectable conditions
    \end{itemize}
    \item Has debug tools to understand the inferred model
  \end{itemize}
\end{frame}

\begin{frame}
  \frametitle{Yeager's API Fits On A Notecard}
  \begin{itemize}
    \item \texttt{import yeager}
    \item \texttt{@yeager.state\_transition(from, to)}
    \item \texttt{yeager.walk()}
    \item Tweak: \texttt{yeager.add\_state\_to\_blacklist()}, \texttt{yeager.add\_transition\_to\_blacklist()}, \texttt{yeager.remove\_state\_from\_blacklist()}, \texttt{yeager.remove\_transition\_from\_blacklist()}, and \texttt{yeager.set\_edge\_weight()}
    \item Debug: \texttt{yeager.enumerate\_transitions()}, \texttt{yeager.reachable\_states()}, \texttt{yeager.orphaned\_states()}
  \end{itemize}
\end{frame}

\begin{frame}[fragile]
  \frametitle{Write a Function}
  \begin{lstlisting}
  def login(driver):
    from pages.login import LoginPage
    lp = LoginPage(driver)
    lp.log_in_correctly(USERNAME, PASSWORD)
  \end{lstlisting}
\end{frame}

\begin{frame}[fragile]
  \frametitle{Annotate the State Transition}
  \begin{lstlisting}
  @yeager.state_transition("login", "dashboard")
  def login(driver):
    from pages.login import LoginPage
    lp = LoginPage(driver)
    lp.log_in(USERNAME, PASSWORD)
  \end{lstlisting}
\end{frame}

\begin{frame}
  \frametitle{Debug Yeager Models}
  \begin{itemize}
    \item Using \texttt{enumerate\_transitions} function as show in \texttt{enumerate\_transitions.py} %DEMO
    \item Using \texttt{orphaned\_states} \& \texttt{reachable\_states} functions as shown in \texttt{orphaned\_states.py} \& \texttt{reachable\_states.py} %DEMO
  \end{itemize}
\end{frame}

\begin{frame}
  \frametitle{Plan And Execute A Test Run}
  \begin{itemize}
    \item \texttt{yeager.walk()}
    \item \texttt{yeager.walk(50)}
    \item \texttt{yeager.walk(exit\_state="state-to-exit-on")}
    \item In development: after some visitation goal
  \end{itemize}
\end{frame}

%----------------------------
\subsection{Yeager In Action}

\begin{frame}
  \frametitle{Intuitive States in a CRM}
  \begin{itemize}
    \item login page
    \item dashboard
    \item contacts list
    \item looking at a contact
    \item editing a contact
    \item logging a phone call or meeting with a contact
    \item writing in the journal
    \item etc.
  \end{itemize}
\end{frame}

\begin{frame}
  \frametitle{States Necessitate Transitions}
  \begin{itemize}
    \item Filling in the login form transitions from the login page to the dashboard
    \item Clicking a contact in the contacts list transitions to the viewing-a-contact state
    \item etc.
  \end{itemize}
\end{frame}

\begin{frame}
  \frametitle{Write Some Glue and Go}
  Break up existing test code into logical state-transitioning units. Then for each unit:
  \begin{itemize}
    \item extract a relatively stateless function from it.
    \item annotate whatever transition that function triggers.
  \end{itemize}
\end{frame}

\begin{frame}
  \frametitle{A Note on ``Relative Statelessness''}
  \begin{itemize}
    \item This will vary from tester to tester according to their gumption (and requirements).
    \item It is reasonable for a test function to require a shared webdriver so in a web test.
    \item It might be reasonable for a test function to require a list of all the contact names put into the system so far.
    \item It is unreasonable for a test function to require a memoizing key-value store with hundreds or thousands of entries.
    \item All extra arguments passed to \texttt{walk} are forwarded to test functions.
    \item Mutable arguments can be modified and these modifications persist across execution.
\end{itemize}

\end{frame}

\begin{frame}
  \frametitle{Visualizing the Simplified State Model}
  It is straightforward to use the Yeager graph inference with graph visualization software. A routine is provided to allow users to visualize with the \texttt{graph\_tool} module, which can further export to the more-standardized graphviz package natively.

  \texttt{python3 visualize\_graph.py} % DEMO
\end{frame}

\begin{frame}
  \frametitle{Take a Walk}
  \begin{itemize}
    \item Execution begins with a call to \texttt{yeager.walk()}
    \item A demo: \texttt{python3 yeager\_test.py} % DEMO
  \end{itemize}
\end{frame}

\begin{frame}
  \frametitle{What It Looks Like The Test Is Going Well}
  \begin{itemize}
    \item No crash
    \item No assertions being tripped
    \item Software appears to be being executed
  \end{itemize}
\end{frame}

\begin{frame}
  \frametitle{What It Looks Like When The Model Is Wrong}
  \begin{itemize}
    \item Crash on an illogical sequence
    \item Example:
    \begin{itemize}
      \item Click ``Create Contact''
      \item Click ``Add this Contact''
      \item Expected: On Contact pages
      \item Actual: On Add Contact Page with an error message about needing to input a name
    \end{itemize}
    \item A suite can generate this fault: \texttt{yeager\_bad\_model\_test.py} % DEMO
  \end{itemize}
\end{frame}

\begin{frame}
  \frametitle{What It Looks Like When The Software Is Wrong}
  \begin{itemize}
    \item Crash on a perfectly logical sequence
    \item Example:
    \begin{itemize}
      \item Open a contact
      \item Click ``Add Reminder''
      \item Fill in a date
      \item Fill in a title
      \item Check the ``Remind me about this just once'' box
      \item Click the save button
      \item Expected: On the contact's page, with a new reminder
      \item Actual: On a 500 internal server error page
    \end{itemize}
    \item \url{https://github.com/monicahq/monica/issues/326}
  \end{itemize}
\end{frame}

%-------------------------------
\subsection{The Case for Yeager}

\begin{frame}
  \frametitle{SSM-Based LSRT}
  \begin{itemize}
    \item Benefits of LSRT by building on existing test automation investment, and exposing behavior under arbitrarily long test sequences
    \item Benefits of FSM modeling by thoroghly exploring the system, as well as providing valuable insight into the construction of the system
  \end{itemize}
\end{frame}

\begin{frame}
  \frametitle{Quick To Implement}
  \begin{itemize}
    \item Tests can be built as quickly as the tester can write Python.
    \item Tests benefit from good engineering practices elsewhere in the testing effort.
    \item Tests can focus on areas of the system under inspection, an incomplete model is still valuable unlike in FSMs.
  \end{itemize}
\end{frame}

\begin{frame}
  \frametitle{Selective Detail}
  \begin{itemize}
    \item Testers can hammer small details like keystrokes into a textbox or focus only on big-picture program flow.
    \item Testers make as many or few assertions as they wish.
    \item Testers can control the flow of their walks depending on the testing context.
  \end{itemize}
\end{frame}

\begin{frame}
  \frametitle{Another Tool To Fight Failures of Imagination}
  \begin{itemize}
    \item In some contexts, against some classes of bug, with sufficient resources
    \item Good use of idle systems
    \item Fill in a gap on the HiVAT family tree
  \end{itemize}
\end{frame}
