%=====================================
\section{High Volume Automated Testing}

\begin{frame}
  \frametitle{What Is High Volume Test Automation (HiVAT)?}
  Tests that algorithmically generate, execute, and evaluate the results of arbitrarily many test actions on a system, in such volume as to:\citep{KanerHivatOverview}
  \begin{enumerate}
  \item Exceed the volume a reasonable testing staff could do manually.
  \item Expose behaviors of the system not normally exposed during traditional testing techniques.
  \item Simulate use and abuse of the system more realistically and dynamically than would be attainable through traditional techniques.
  \item Generate test scenarios that are not outside the realm of possibility or even probability due to the high-availability nature of modern software systems.
\end{enumerate}
\end{frame}

%------------------
\subsection{Anatomy}

\begin{frame}
  \frametitle{Generators}
  \begin{itemize}
    \item How test cases are generated
    \item How the system is driven
    \item An engineering consideration
  \end{itemize}
\end{frame}

\begin{frame}
  \frametitle{Interface}
  \begin{itemize}
    \item Black box or white box
    \item Shades of grey, maybe hitting a private REST service instead of the UI directly
    \item A consideration of engineering and testing goals
  \end{itemize}
  \citep{HoffmanTradeoffs}
\end{frame}

\begin{frame}
  \frametitle{Oracle}
  \begin{itemize}
    \item How to programmatically determine correctness of generated tests
    \item Comparison of some sort
    \begin{itemize}
      \item To assertions in previously written code
      \item To expectations from a formal Finite State Machine
      \item To a previous version of the system
      \item To a competitor's system
      \item To systemic expectations, like not crashing
      \item Room for research here
    \end{itemize}
    \item A consideration of engineering and testing goals
  \end{itemize}
\end{frame}

\begin{frame}
  \frametitle{Loggers and Diagnostics}
  \begin{itemize}
    \item Keeping track of test trace
    \item Keeping track of system health during test
    \item Possibly characterizing system degradation
    \item A consideration of testing goals
  \end{itemize}
\end{frame}

\begin{frame}
  \frametitle{Context}
  \begin{itemize}
    \item Testing objectives regardless of engineering
    \begin{itemize}
      \item Surveying the system for new bugs
      \item Determining system resillience through abuse
      \item Cornering hard-to-replicate bugs in suspect modules
      \item Characterizing system resource consuption over time
    \end{itemize}
  \end{itemize}
\end{frame}

\begin{frame}
  \frametitle{Scalability}
  \begin{itemize}
    \item How volume in these tests is generated
    \begin{itemize}
      \item A single, long-running thread
      \item A cluster of many threads
      \item A swarm of many cheap cloud servers \citep{parveen2010migrate}
      \item A virtualization service testing a breadth of configurations
    \end{itemize}
    \item A consideration of the testing context and engineering constraints
  \end{itemize}
\end{frame}

%------------------
\subsection{History}

\begin{frame}
  \frametitle{Purported Inventors}
  \begin{itemize}
    \item HP's ``evil''
    \begin{itemize}
      \item Oldest in my literature review from 1966
    \end{itemize}
    \item TI
    \item Bell
    \item AT\&T
    \item Microsoft
    \item Telenova
    \item Rohm
    \item FAA contractors
    \item Automotive industry
    \item \citet{miller1989TR830} with the Fuzz Tester
    \begin{itemize}
      \item First from academia, 1989 technical report and 1990 article.
    \end{itemize}
  \end{itemize}
\end{frame}

\begin{frame}
  \frametitle{Industrial Inventors Are Reticent To Publish}
  \begin{itemize}
    \item HiVAT is perceived as a competititve advantage
    \item Disclosing these practices would expose testers to risk of termination or legal retaliation
    \item Swept away as part of efforts to minimize maintenance-related tasks
  \end{itemize}
\end{frame}

%------------------------
\subsection{Family Tree}

\begin{frame}
  \frametitle{LSRT: Long Sequence Regression Testing}
  \begin{itemize}
    \item Accomplished by modifying exisiting test suites
    \item Set tests to run continuously
    \item Remove cleanup between test runs
  \end{itemize}
\end{frame}

\begin{frame}
  \frametitle{State Model Testing}
  \begin{itemize}
    \item Build a detailed Finite State machine
    \item Algorithmically exercise the machine to generate testable theorems about the system
  \end{itemize}
  \citep{lee1996principles}
\end{frame}

\begin{frame}
  \frametitle{Exhaustive Testing}
  \begin{itemize}
    \item Lower level
    \item Test every single possible parameter value to a function
    \item Needs another implementation for an oracle
    \item Gets prohibitively slow for multiple parameters
    \item Analysis, using slices for instance \citep{gallagher1991using}, can prove parameter independence and eliminate the need to test combinations of parameters
  \end{itemize}
\end{frame}

\begin{frame}
\frametitle{A Tale Of Two Exhaustive Tests}
\begin{columns}[c]
  \column{.5\textwidth}
    \textbf{\citet{hoffman2003Exhausting}}
    \begin{itemize}
      \item Suspected a trig function of bugs
      \item Used another implementation
      \item Fed both functions every number in the range of a 32 bit float
      \item Found two errors in a few minutes
    \end{itemize}
  \column{.5\textwidth} % Right column and width
    \textbf{\citet{dawsonFourBillion}}
    \begin{itemize}
      \item Suspected a trig function of bugs
      \item Used another implementation
      \item Fed both functions every number in the range of a 32 bit float
      \item Found one error 826k times in about 90 seconds
    \end{itemize}
\end{columns}
\end{frame}

\begin{frame}
  \frametitle{Fuzz Testing}
  \begin{itemize}
    \item Miller's tool generates streams of random bytes and feeds them as input to UNIX command line utilities. \citep{miller1990empirical}
    \item A test fails if the program crashes.
    \item Fuzz testing has grown into a diverse family of subtechniques, popular among security researchers.
  \end{itemize}
\end{frame}

\begin{frame}
  \frametitle{Load/Performance Testing}
  \begin{itemize}
    \item API tests put into a massive thread pool
    \item The ``accepted'' way to verify many users won't crash a system
    \item Popular tool in this family: Locust \citep{heymanlocust}
  \end{itemize}
\end{frame}

\begin{frame}
  \frametitle{Testing In Production}
  \begin{itemize}
    \item A practice at Microsoft \citep{AndrewsHeal}
    \item Candidate builds of Bing fed actual user input
    \item Output compared to current build
    \item Enables automated, staged deployments
  \end{itemize}
\end{frame}

\begin{frame}
  \frametitle{A/B Testing}
  \begin{itemize}
    \item Marketing practice
    \item Release candidate revisions to a subset of users and monitor for desireable behavior
    \item Promote the most effective revision to general availability
    \item Email marketing, site homepages, search engine ads, news stories
  \end{itemize}
  \citep{HBR2017ABTest}
\end{frame}

%-------------------------------
\subsection{The Case for Yeager}

\begin{frame}
  \frametitle{Model-Based LSRT}
  \begin{itemize}
    \item Benefits of LSRT by building on existing test automation investment, and exposing behavior under arbitrarily long test sequences
    \item Benefits of FSM modeling by thoroghly exploring the system, as well as providing valuable insight into the construction of the system
  \end{itemize}
\end{frame}

\begin{frame}
  \frametitle{Quick To Implement}
  \begin{itemize}
    \item Tests can be built as quickly as the tester can write Python.
    \item Tests benefit from good engineering practices elsewhere in the testing effort.
    \item Tests can focus on areas of the system under inspection, an incomplete model is still valuable unlike in FSMs.
  \end{itemize}
\end{frame}

\begin{frame}
  \frametitle{Selective Detail}
  \begin{itemize}
    \item Testers can hammer small details like keystrokes into a textbox or focus only on big-picture program flow.
    \item Testers make as many or few assertions as they wish.
    \item Testers can control the flow of their walks depending on the testing context.
  \end{itemize}
\end{frame}
