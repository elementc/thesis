%==========================
\section{Automated Testing}

\begin{frame}
  \frametitle{Why Automate?}
  \begin{itemize}
    \item Save time.
    \item Save money.
    \item Computers don't get bored.
    \begin{itemize}
      \item Testing is boring work.
    \end{itemize}
    \item Doesn't overlook test cases.
  \end{itemize}
\end{frame}

\begin{frame}
  \frametitle{How Do You Automate?}
  \begin{itemize}
    \item Write functions that exercise the system under test.
    \item Put these functions in a format that can be consumed by a test runner.
    \item Call test runner.
    \item Interpret test runner's output.
  \end{itemize}
\end{frame}

\begin{frame}
  \frametitle{Kinds of Test Automation}
  \begin{itemize}
    \item Unit- Verify individual code functions work as expected.
    \item Integration- Verify modules are working together.
    \item Regression- Verify stuff that worked isn't broken and stuff that's broken before haven't broken again.
    \item Functional- Validate that system as a whole conforms to requirements (eg, works to a user's eye).
    \item Many others- see CSE3411 \& CSE4415.
  \end{itemize}
\end{frame}

%-----------------------
\subsection{Technologies}

\begin{frame}
  \frametitle{Languages}
  \begin{itemize}
    \item *Unit frameworks (CUnit, JUnit, CppUnit, etc.) enable the practice, though they're useful for far more than unit testing.
    \item Testers are usually, unintuitively, \textbf{less} trained as programmers.
    \item Consequently, they prefer ``easier'' scripting languages like Python or Ruby.
    \item This discussion will center around Python. All of it can happen in Ruby.
  \end{itemize}
\end{frame}

\begin{frame}
  \frametitle{Frameworks}
  \begin{itemize}
    \item Include a suite of assertion convenience methods, logging/reporting facilities, and a runner.
    \item Python: \texttt{unittest}, \texttt{nose}, \texttt{pytest}.
    \item \texttt{unittest} is in the Python Standard Library, we'll use it.
  \end{itemize}
\end{frame}

\begin{frame}
  \frametitle{Glass Box Testing}
  \begin{itemize}
    \item Test code interacts directly with the program's source.
    \item Can probe quite deeply.
    \item Use mock interfaces shims to accomplish testing goals.
    \item Unit testing and integration testing are automated through Glass Box methods.
  \end{itemize}
\end{frame}

\begin{frame}
  \frametitle{Black Box Testing}
  \begin{itemize}
    \item Test code interacts with the user (or some other non-transparent) interface into the running program.
    \item Use external toolkits like Selenium to enable driving user interfaces.
    \item Usually in a special test environment but otherwise the unmodified software.
    \item Regression testing and functional testing are automated through Black Box methods.
  \end{itemize}
\end{frame}

\begin{frame}
  \frametitle{Selenium}
  \begin{itemize}
    \item Programmatic control of web browsers for testing and other automation.
    \item Driver class allows navigation (get this URL) and document queries (get this node for me to read from or click on or type into).
    \item Node class allows interaction (click here, type this), and data retrieval (What's the text body? What' site does this form POST to?) and limited Driver-like queries for children.
  \end{itemize}
\end{frame}

\begin{frame}
  \frametitle{HTML (summary)}
  \begin{itemize}
    \item XML- based documents for the web.
    \item Tree-structured.
    \item Nodes have properties, including text, in addition to children.
  \end{itemize}
\end{frame}

\begin{frame}
  \frametitle{CSS (summary)}
  \begin{itemize}
    \item Language for styling HTML documents.
    \item Format- \texttt{selector}: \texttt{rule};
    \item Selectors: strings that identify one, many, or none of the nodes in an HTML document.
    \item Rules: Specific styling rules to apply to each node matched by preceeding rule.
  \end{itemize}
\end{frame}

%-----------------------------------------
\subsection{System Under Test: Monica CRM}
\begin{frame}
  \frametitle{Monica: A Personal CRM}
  \begin{itemize}
    \item Open-Source.
    \item Life-tracker.
    \item Friend-keeper.
    \item Journal.
    \item In the cloud.
  \end{itemize}
\end{frame}

\begin{frame}
  \frametitle{Contacts Book On Steroids}
  \begin{itemize}
    \item We've all seen a contact list as an example in a database course.
    \item Monica extends that to the extreme. Per-contact notebooks, relationship tracking (How many wives did he have?), reminders, etc.
    \item Screenshots TBD
  \end{itemize}
\end{frame}

%----------------------------------
\subsection{Patterns and Practices}

\begin{frame}
  \frametitle{Page Object Modeling}
  \begin{itemize}
    \item Each page on a site corresponds to a Python class.
    \item Fields or text spots on pages get getters and setters.
    \item Clickable things get click() functions.
    \begin{itemize}
      \item If the click should transition to a new page, construct and return that new page's class.
    \end{itemize}
    \item In class consturctors, assert things that are constant about that page. Username in header? Some error message not in the body? Standard controls are present?
  \end{itemize}
\end{frame}

\begin{frame}
  \frametitle{How Web Test Suites Come Together}
  \begin{itemize}
    \item Build all the page objects and put them in /pages/.
    \item Write step-by-step test plan as comments in the body of a function in the runner's format.
    \item Translate english steps into python code.
  \end{itemize}
\end{frame}

\begin{frame}
  \frametitle{Running Tests}
  \begin{itemize}
    \item Same as running any other script.
    \item \texttt{python3 test\_contacts.py}
    \item Some frameworks have a multi-script runner.
    \item \texttt{python3 -m unittest}
  \end{itemize}
\end{frame}
