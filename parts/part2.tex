%========================================
\section{Long Sequence Testing in Yeager}

\begin{frame}
  \frametitle{'Bugs' That Traditional Testing Finds}
  \begin{itemize}
    \item Known bugs, whether previously fixed or bugs that are defended against
    \item Unfinished features
    \begin{itemize}
      \item Write the tests before you write the feature.
    \end{itemize}
    \item Clear and obvious program faults
    \begin{itemize}
      \item Obvious to the computer
      \item Crashes, for instance
      \item Nonzero return codes
    \end{itemize}
  \end{itemize}
\end{frame}

\begin{frame}
  \frametitle{What Traditional Testing Does Not Find}
  \begin{itemize}
    \item Faults the tester did not think to test for
    \item Faults that are not obvious
    \item Faults the tester deems improbable
  \end{itemize}
\end{frame}

\begin{frame}
  \frametitle{How To Find What Traditional Testing Does Not Find}
  \begin{itemize}
    \item All the bugs missed are failures of imagination.
    \begin{itemize}
      \item If a scenario can be imagined, a test can be written for it.
    \end{itemize}
    \item Computers are really bad at imagining, too, but are passable at rolling dice.
  \end{itemize}
\end{frame}

\begin{frame}
  \frametitle{Examples of The Bugs We Want To Find}
  \begin{itemize}
    \item Digital phone system that crashes when the 22nd line is put on hold
    \item Flaky text editor that has been running for months on a grad student's laptop
    \item System that buckles when 200k users log on at the start of a workday
    \item Other ``hard to reproduce'' failures
  \end{itemize}
\end{frame}

%--------------------------------------
\subsection{Software as a State Machine}
\begin{frame}
  \frametitle{Software Is A Finite State Machine}
  \begin{itemize}
    \item Software can be represented as a machine with states, state transitions, inputs, outputs, and other tuples.
    \item FSM exactly describes the software's behavior
    \item Technique is popular in EE and for testing protocols
  \end{itemize}
\end{frame}

\begin{frame}
  \frametitle{Testers Write Based On The System's States}
  \begin{itemize}
    \item Page Object Model testing pattern emulates the system's underlying state model, and includes state transitions.
    \item Implied state model is significantly simplified compared to a formal FSM specification.
    \item POM provides a detailed look at how the system is built.
  \end{itemize}
\end{frame}

\begin{frame}
  \frametitle{State Models Can Help Us Plan New Tests}
  \begin{itemize}
    \item Given a printout of a state model, one can trace a pen along the model and plan a new test sequence.
    \item What parts of the SUT are tested and what parts are not yet tested becomes obvious.
  \end{itemize}
\end{frame}

\begin{frame}
  \frametitle{Context: What Simplified State Models Don't Capture}
  \begin{itemize}
    \item Input typed into the program
    \item Data the program read from some external source
    \item Overheating CPUs, cosmic rays, etc.
  \end{itemize}
\end{frame}

\begin{frame}
  \frametitle Simplified State Models Can Be Represented As Directed Multigraphs
  \begin{itemize}
    \item System states are vertexes, or nodes.
    \item Test functions are edges, connecting an in-node to an out-node.
    \item Each edge connects one in-node to one out-node, however
    \begin{itemize}
      \item a given function might work as a transition to an out-node from multiple compatible in-nodes.
      \item This behavior is a byproduct of convenicence features in the software under test, like having a logout button on every page.
      \item For brevity's sake, treat a list of in-nodes on an edge's definition as a separate edge definition for each listed in-node.
    \end{itemize}
  \end{itemize}
\end{frame}

\begin{frame}
  \frametitle{Random Walks: Generating New Test Plans Automatically}
  Given one of these simplified state models represented as a graph, and a source of random numbers, automatically generating test plans is straightforward.
  \begin{itemize}
    \item For a given node, the current state, from the set of nodes
    \item Gather all of the edges, the transition functions, which have that state as their from-node
    \item Select one of these gathered functions at random and execute it
    \item The selected function's to-node becomes the new current state
    \item Repeat until some planned condition is met or execution of a selected function is not possible
  \end{itemize}
\end{frame}

\begin{frame}
  \frametitle{What Bugs Look Like From A Modeling Perspective}
  \begin{itemize}
    \item Bugs manifest as nodes which the model says should be reachable, but execution cannot successfully reach.
    \item Such occurrences might be bugs in the software.
    \item Such occurrences might be bugs in the tester's model.
  \end{itemize}
\end{frame}

\begin{frame}
  \frametitle{Prior Art: Model Based Testing}
  \begin{itemize}
    \item Jonathan Jacky, in Radiation Oncology, of the University of Washington, made an excellent Python model-based tester called PyModel.
    \item PyModel consumes a handcrafted model.
    \item PyModel can emit a test plan that covers the whole model.
    \item PyModel can emit a test plan that takes a random, should-be valid walk of the software under test.
  \end{itemize}
\end{frame}

\begin{frame}
  \frametitle{Weaknesses in PyModel}
  \begin{itemize}
    \item PyModel requires a handcrafted model in a finicky domain-specific language.
    \begin{itemize}
      \item Not Plain Old Python.
    \end{itemize}
    \item PyModel is difficult to connect to test execution.
    \item PyModel requires a lot of time to get running.
  \end{itemize}
\end{frame}

%----------------
\subsection{Usage}

\begin{frame}
  \frametitle{What Is Yeager?}
  \begin{itemize}
    \item Python version 3 module
    \item Annotate funtions indicating that they cause a state transition.
    \item Infers a state model
    \item Can take a random walk on that model
    \begin{itemize}
      \item Can terminate random walks under selectable conditions
    \end{itemize}
    \item Has debug tools to understand the inferred model
  \end{itemize}
\end{frame}

\begin{frame}
  \frametitle{Yeager's API Fits On A Notecard}
  \begin{itemize}
    \item \texttt{import yeager}
    \item \texttt{@yeager.state\_transition(from, to)}
    \item \texttt{yeager.walk()}
    \item Tweak: \texttt{yeager.add\_state\_to\_blacklist()}, \texttt{yeager.add\_transition\_to\_blacklist()}, \texttt{yeager.remove\_state\_from\_blacklist()}, \texttt{yeager.remove\_transition\_from\_blacklist()}, and \texttt{yeager.set\_edge\_weight()}
    \item Debug: \texttt{yeager.enumerate\_transitions()}, \texttt{yeager.reachable\_states()}, \texttt{yeager.orphaned\_states()}
  \end{itemize}
\end{frame}

\begin{frame}[fragile]
  \frametitle{Write a Function}
  \begin{lstlisting}
  def login(driver):
    from pages.login import LoginPage
    lp = LoginPage(driver)
    lp.log_in_correctly(USERNAME, PASSWORD)
  \end{lstlisting}
\end{frame}

\begin{frame}[fragile]
  \frametitle{Annotate the State Transition}
  \begin{lstlisting}
  @yeager.state_transition("login", "dashboard")
  def login(driver):
    from pages.login import LoginPage
    lp = LoginPage(driver)
    lp.log_in(USERNAME, PASSWORD)
  \end{lstlisting}
\end{frame}

\begin{frame}
  \frametitle{Debug Yeager Models}
  \begin{itemize}
    \item Using \texttt{enumerate\_transitions} function %DEMO
    \item Using \texttt{orphaned\_states} \& \texttt{reachable\_states} functions %DEMO
  \end{itemize}
\end{frame}

\begin{frame}
  \frametitle{Plan A Test Run}
  \begin{itemize}
    \item \texttt{yeager.walk()}
    \item \texttt{yeager.walk(50)}
    \item \texttt{yeager.walk(exit\_state="state-to-exit-on")}
    \item In development: after some visitation goal
  \end{itemize}
\end{frame}

\begin{frame}
  \frametitle{Run It}
  \begin{itemize}
    \item \texttt{python3 yeager\_test.py} % DEMO
  \end{itemize}
\end{frame}

%----------------------------
\subsection{Yeager In Action}

\begin{frame}
  \frametitle{Test Monica With Yeager}
  \begin{itemize}
    \item Have a robust suite of Page Object Models
    \item Intuitive and meaningful system
    \item Public service
  \end{itemize}
\end{frame}

\begin{frame}
  \frametitle{Intuitive States of Monica}
  \begin{itemize}
    \item login page
    \item dashboard
    \item contacts list
    \item looking at a contact
    \item editing a contact
    \item logging a phone call or meeting with a contact
    \item writing in the journal
    \item etc.
  \end{itemize}
\end{frame}

\begin{frame}
  \frametitle{States Necessitate Transitions}
  \begin{itemize}
    \item Filling in the login form transitions from the login page to the dashboard
    \item Clicking a contact in the contacts list transitions to the viewing-a-contact state
  \end{itemize}
\end{frame}

\begin{frame}
  \frametitle{Use Existing Page Object Models As A Guide}
  \begin{itemize}
    \item Emulates the Page Object Models' structure
    \item States are pages
    \item Methods are state transitions
    \begin{itemize}
      \item Some transitions can be loopbacks
    \end{itemize}
  \end{itemize}
\end{frame}

\begin{frame}
  \frametitle{Write Some Glue and Go}
  For each method in the page object models:
  \begin{itemize}
    \item create a relatively stateless function that calls it.
    \item annotate any state transition that function triggers.
  \end{itemize}
\end{frame}

\begin{frame}
  \frametitle{``Relative Statelessness''}
  \begin{itemize}
    \item This will vary from tester to tester according to their gumption.
    \item It's reasonable for a test function to require a shared webdriver so page objects can be used.
    \item It might be reasonable for a test function to require a list of all the Contact names put into the system so far.
    \item It's unreasonable for a test function to require a memoizing key-value store with hundreds or thousands of entries.
\end{itemize}

\end{frame}

\begin{frame}
  \frametitle{Example Suite's Model}
  % TODO: PICTURE AND DEMO
\end{frame}

\begin{frame}
  \frametitle{Give It A Run}
  \begin{itemize}
    \item Execution begins with a call to \texttt{yeager.walk()}
  \end{itemize}
\end{frame}

\begin{frame}
  \frametitle{What It Looks Like When Everything Is Good}
  \begin{itemize}
    \item No crash
    \item No assertions being tripped
    \item Software appears to be being executed
  \end{itemize}
\end{frame}

\begin{frame}
  \frametitle{What It Looks Like When The Model Is Wrong}
  \begin{itemize}
    \item Crash on an illogical sequence
    \item Example:
    \begin{itemize}
      \item Click "Create Contact"
      \item Click "Add this Contact"
      \item Expected: On Contact pages
      \item Actual: On Add Contact Page with an error message about needing to input a name
    \end{itemize}
  \end{itemize}
\end{frame}

\begin{frame}
  \frametitle{What It Looks Like When The Software Is Wrong}
  \begin{itemize}
    \item Crash on a perfectly logical sequence.
    \item Example:
    \begin{itemize}
      \item Open a contact
      \item Click "Add Reminder"
      \item Fill in a date
      \item Fill in a title
      \item Check the ``Remind me about this just once'' box
      \item Click the save button
      \item Expected: On the contact's page, with a new reminder
      \item Actual: On a 500 internal server error page
    \end{itemize}
    \item \url{https://github.com/monicahq/monica/issues/326}
  \end{itemize}
\end{frame}
